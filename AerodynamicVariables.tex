\documentclass[a4paper, 12pt]{report}

\usepackage[russian]{babel}
\usepackage[utf8]{inputenc}
\usepackage[T1]{fontenc}
\usepackage{amsmath, amssymb, amsfonts}
\usepackage[Symbolsmallscale]{upgreek}

\renewcommand{\alpha}{\upalpha}
\renewcommand{\delta}{\updelta}
\renewcommand{\lambda}{\uplambda}
\renewcommand{\zeta}{\upzeta}

\newcommand{\Mach}{\mathrm{M}}
\newcommand{\rut}{\text{т}}

\begin{document}
	\begin{center}
		\textbf{Общие}
		\begin{tabular}{c c p{8cm}}
			Обозначение & Имя переменной & Название \\ \hline
			$ \Mach  $  & M              & число Маха \\ \hline
			$ \alpha $  & alpha          & угол атаки \\ \hline
			$ \delta $  & delta          & угол отклонения руля \\ \hline		
		\end{tabular}
		\bigbreak
		
		\textbf{Фюзеляж}
		\begin{tabular}{c c p{8cm}}
			Обозначение             & Имя переменной  & Название \\ \hline
			$ S_\text{ф}    $       & S\_f            & площадь миделя \\ \hline
			$ L_\text{ф}    $       & L\_f            & длина фюзеляжа \\ \hline
			$ \lambda_\text{нос} $  & lambda\_nos     & удлинение носовой части \\ \hline		
		\end{tabular}
		\bigbreak
		
		\textbf{Несущие поверхности}
		\begin{tabular}{c c p{8cm}}
			Обозначение             & Имя переменной  & Название \\ \hline
			$ S_\text{к}   $        & S\_c            & площадь консоли \\ \hline
			$ l_\text{к}    $       & l\_c            & размах консолей \\ \hline
			$ \zeta_\text{к} $      & zeta\_c         & обратное сужение консолей \\ \hline		
		\end{tabular}
		\bigbreak
		
		\textbf{Летательный аппарат}
		\begin{tabular}{c c p{8cm}}
			Обозначение             & Имя переменной  & Название \\ \hline
			$ c_{x_a}   $           & c\_xa           & коэффициент лобового сопротивления \\ \hline
			$ S       $             & S               & характерная площадь \\ \hline
			$ x_I $                 & x\_I            & расстояние от носа до носка бортовой хорды первой несущей поверхности \\ \hline		
		\end{tabular}
	\end{center}

\end{document}